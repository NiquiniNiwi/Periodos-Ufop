\chapter{Discussão} \label{discussao}
Com o passar do anos a plataforma cresceu cada vez mais, sendo que o maior pico foi de março para abril de 2020, com aproxamadamente 573 milhões de horas visualizadas a mais em abril. Também é possivel visualizar uma constancia desde o começo de 2018 até o final do primeiro semestre de 2019, mostrando que a plataforma tem um publico constante a muito tempo e que isso pode se repetir novamente num periodo pos pandemia. Como um ultimo ponto a tratatar das visualizações, é possivel perceber que em periodos de ferias escolares, janeiro, julho e agosto, tem picos de visualização maiores do que os outros meses durante o ano.


Ja sobre as catergorias, é possivel ver que alguns tem maior constancia, como "league of legends", "dota 2" e "minecraft", ja outros, como "Valorant" e  "COD: MW", tem picos e depois tendem a cair gradativamente, sendo mais visivel no caso do "Valorant" aonde em seu lançamento ele foi a categoria mais vista, mas nos meses seguintes so ouve queda. Mas existe uma categoria que se beneficiou de uma maior diveridade de publico e vem crescendo constantemente, "Just chatting" vem como uma categoria emergente desde seu surgimento em setembro de 2018 e que explodiu desde o começo de 2020, quando começou-se a congitar uma quarentena.


