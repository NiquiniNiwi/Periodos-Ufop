\chapter{Revisão de Literatura} \label{RevisaoBibliografica}

Neste capítulo, deve ser apresentada uma contextualização da sua pesquisa com um resumo das discussões já feitas por outros autores sobre o assunto abordado e os conceitos principais relativos ao tema. O nome deste capítulo  de \textbf{Revisão Bibliográfica}, \textbf{Revisão de Literatura} ou \textbf{Embasamento Teórico} deve ser acordado com seu orientador.

A revisão de literatura é a base que sustenta qualquer pesquisa científica e é indispensável para a delimitação do problema em um projeto de pesquisa,  para obter uma ideia precisa sobre o estado atual dos conhecimentos sobre um tema, sobre suas lacunas e sobre a contribuição da investigação para o desenvolvimento do conhecimento \cite{marconi2003}. 

Para a escrita deste capítulo, as citações e referências devem estar de acordo com a norma \cite{NBR6023:2002}, que destina-se a orientar a preparação e compilação das  referências bibliográficas de todo o documento.

\section{Fundamentação Teórica}

Nesta seção, devem ser fornecidas uma fundamentação teórica e explicações prévias acerca dos conceitos discutidos ao longo texto. É importante situar o leitor (considerando que este seja leigo no assunto) sobre os termos usados com o intuito de prepará-lo para a leitura dos capítulos subsequentes~\cite{Rampazzo2005,Carvalho1989}.

\section{Trabalhos Relacionados}

Nesta seção, devem ser descritos os principais trabalhos realizados por outros autores sobre a temática escolhida para ser desenvolvida, apresentando os conceitos mais importantes, justificativas e características sobre o tema, do ponto de vista da análise feita pelos autores, além das principais limitações e contribuições dos trabalhos apresentados.  Finalize a seção realizando uma análise comparativa entre o seu trabalho e os trabalhos correlatos, destacando as semelhanças e as diferenças identificadas.








